\begin{resumo}[Abstract]
 \begin{otherlanguage*}{english}
Kidney cancer is a public health problem that affects thousands of people worldwide. The precise segmentation of kidneys and kidney tumors can help medical specialists to diagnose diseases and improve treatment planning, which is highly required in clinical practice. However, because of the heterogeneity of kidneys and kidney tumors, manually segmenting is a time-consuming process and subject to variability among specialists. Because of this hard work, computational techniques, such as convolutional neural networks (CNNs), have become popular in automatic medical image segmentation tasks. Three-dimensional (3D) networks have a high segmentation capacity, but they are complex and have high computational costs. Thus, two-dimensional networks are the most used owing to the relatively low memory consumption, but they do not exploit 3D features. Therefore, in this thesis, 2.5D networks, that balances memory consumption and model complexity, are proposed to doctors specialized in the detection of kidneys and kidney tumors in computed tomography (CT). These networks are inserted in a proposed method organized in four steps: (1) image base pre-processing; (2) initial segmentation of kidneys and kidney tumors using ResUNet 2.5D and DeepLabv3+ 2.5D models, respectively; (3) reconstruction of kidney tumors using binary operation; and (4) reduction of false positives using image processing techniques. The proposed method was evaluated in 210 CTs from the KiTS19 image base. In the segmentation of the kidneys, it presented 97.45\% of Dice, 95.05\% of Jaccard, 99.95\% of accuracy, 98.44\% of sensitivity and 99.96\% of specificity. In the segmentation of renal tumors, 84.06\% of Dice, 75.04\% of Jaccard, 99.94\% accuracy, 88,33\% sensitivity and 99.95\% specificity were obtained. Overall, the results provide strong evidence that the proposed method is a powerful tool to help diagnose the disease.

\textbf{Keywords}: Kidney cancer, Kidney segmentation, Kidney tumor segmentation, Convolutional neural networks, Deep learning, Computed tomography, Medical images.
 \end{otherlanguage*}
\end{resumo}