\chapter{Trabalhos Relacionados}
\label{cap:trabalhos-relacionados}
\phantom{0}

Na literatura existem estudos que tratam do mesmo problema abordado nesta proposta de tese, ou seja, métodos desenvolvidos para segmentação de rins e tumores renais em imagens de TC. As subseções a seguir apresentam resumos de alguns trabalhos relacionados, que se dividem em: segmentação de rins, segmentação de tumores renais, e segmentação de rins e tumores renais.

\section{Segmentação de Rins}
\label{sec:trabalhos-relacionados-rins}

Nos últimos anos, a comunidade científica publicou vários métodos para segmentação de rins em TC. \citeonline{khalifa2011new} descrevem uma abordagem 3D para segmentação renal usando um modelo deformável baseado em \textit{Level Set}. A abordagem proposta foi avaliada nos conjuntos de dados de TC de 29 pacientes, produzindo um Dice de 95\% de rins. Os resultados apresentados indicam que combinar as características das imagens de TC na evolução do conjunto de níveis leva a resultados de segmentação mais precisos.

No trabalho de \citeonline{wolz2013automated} foi apresentado um método automatizado para a segmentação de múltiplos órgãos de tomografias computadorizadas de abdômen. O método baseia-se em um esquema hierárquico de registro e ponderação de atlas. A segmentação final é obtida aplicando um modelo de intensidade aprendido, incorporando conhecimento espacial. Para avaliar o método, foram usados 150 exames de TC, resultando em um Dice de 92,5\%.

No estudo de \citeonline{Zhao2013ContextualIK} uma estrutura de segmentação baseada em fatias é construída para segmentar os rins automaticamente com base na continuidade contextual na sequência de imagens de TC. A estrutura tem quatro etapas principais: segmentação inicial, seleção da segmentação inicial mais confiável e localização e modificação de vazamentos. Um total de 10 pacientes são usados para avaliar a eficácia do método. O resultado final obtido foi de 94,7\% de Dice.

\citeonline{yang2014automatic} propuseram um método automático baseado no registro de imagens multi-atlas. O método é baseado em duas etapas. A primeira etapa consiste no uso de atlas de baixa resolução para uma segmentação inicial. A partir da localização grosseira dos rins, a segunda etapa consiste em cortar os rins esquerdo e direito das imagens originais e alinhar com outro conjunto de imagens de atlas de alta resolução para obter os resultados finais. Para avaliar o método, seis resultados de segmentação de rins com tumores renais grandes foram excluídos porque o tumor alterou significativamente a forma do rim. Finalmente, o método obteve 95,2\% de Dice, usando um conjunto privado de 14 imagens.

No método de \citeonline{khalifa2016kidney} uma estrutura híbrida baseada na integração de modelos geométricos deformáveis e fatoração matricial não-negativa (\textit{Nonnegative Matrix Factorization} - NMF) é usada para segmentação 3D dos rins. Além disso, um modelo de forma é construído usando o conjunto de treinamento e, durante a segmentação, um modelo baseado em aparência é atualizado, levando em consideração as localizações e aparências dos voxels. As interações espaciais são modeladas usando um modelo de campo aleatório de Potts Markov-Gibbs. Um conjunto de 36 exames de TC é usado para avaliar a abordagem proposta e apresentou 96,45\% de Dice. 

O método proposto por \citeonline{skalski2017kidney} é baseado em contorno ativo usando a estrutura \textit{Level Set} com restrições de forma elipsoidal. A solução proposta leva em consideração as informações sobre a região e os termos de contorno, bem como as restrições dedicadas ao formato renal característico. O método proposto foi testado em 10 exames de pacientes com diagnóstico de câncer renal. A média em relação ao Dice foi igual a 86,2\%.

O trabalho de \citeonline{jackson2018deep} desenvolve uma ferramenta automatizada com base em redes neurais convolucionais (\textit{Convolutional Neural Network} - CNNs) 3D capaz de segmentar o rim direito e esquerdo com precisão. Os autores optaram pela abordagem 3D por se mostrar mais adaptável a anormalidades estruturais mais complexas, como os cistos renais. Os experimentos foram realizados em TCs de 113 pacientes, obtendo-se resultados de 91\% para o rim direito e 86\% para o rim esquerdo, ou seja, 88,5\% de Dice médio. O baixo desempenho foi observado em três pacientes com rins císticos.

Outro método é apresentado por \citeonline{mehta2019segmenting}. Os autores avaliam a precisão e a viabilidade de uso das segmentações renais por \textit{crowdsourcing}. 42 TCs, foram rotuladas por 72 usuários na plataforma Robovision AI. As segmentações de \textit{crowdsourcing} e as segmentações rotuladas por especialistas foram usadas individualmente e em conjunto como dados de treinamento para modelos separados da CNN. A validação primária foi realizada comparando as segmentações. O desempenho da CNN atingiu um resultado de 93,2\% de Dice.

%%--novo
\citeonline{9534007} apresentaram um sistema de apoio ao diagnóstico renal. O sistema permite a detecção e segmentação totalmente automática do rim usando um conjunto de 270 imagens de TC. O método proposto tem duas etapas. Na primeira etapa, é realizada a segmentação semântica baseada na rede U-Net e \textit{batch}. Na segunda etapa, um sistema de rastreamento (\textit{middle-bottom, middle-top}) foi usado para preencher a imagem, o que permitiu encontrar o contorno do rim nas fatias restantes da TC. O método desenvolvido atingiu um Dice de 90,63\%.
%De todo o conjunto de todos os 300 casos, 270 foram selecionados. Algumas das imagens tinham uma fase de contraste inadequada; artefatos como implantes de aço e outros eram ilegíveis. 30 teste.

Ao analisar os trabalhos mencionados, observa-se que há desenvolvimento de métodos computacionais para segmentação de rins em imagens de TC. Além disso, é possível verificar que essa tarefa não é algo recente e trivial, ao longo dos anos vários métodos foram elaborados para a segmentação de rins. As abordagens baseadas em contorno ativo~\cite{skalski2017kidney} possuem dependência de pontos iniciais do contorno, parâmetros adequados, não seguem alterações topológicas de objetos e imagens de alta resolução requerem um longo tempo de execução. Além disso, são sensíveis aos estados mínimos locais, sua precisão depende da política de convergência~\cite{bakovs2007active}.

Por sua vez, os métodos deformáveis baseiam-se principalmente no conhecimento ou usam atlas~\cite{wolz2013automated,yang2014automatic,yang2018automatic}. A desvantagem desse tipo de abordagem é que geralmente possui um determinado tamanho para resolver o problema em um conjunto de imagens. Quando novas imagens de resoluções diferentes do atlas são testadas, o método não tem garantia de que funcionará~\cite{SOUZA2019285}. Além disso, técnicas que reduzem a interferência manual de médicos especialistas \cite{mehta2019segmenting} para marcar a região de interesse não são triviais, pois podem ocorrer erros graves, que prejudicariam o modelo e, consequentemente, o diagnóstico médico.

Com o surgimento de \textit{hardwares} cada vez mais sofisticados, começam a surgir trabalhos baseados em aprendizado profundo, que apresentam resultados expressivos em diferentes problemas. Trabalhos como \citeonline{khalifa2011new}, \citeonline{mehta2019segmenting} e \citeonline{9534007} usam redes totalmente conectadas para segmentação de rins. A vantagem de usar essas técnicas é que não há necessidade de uma engenharia de características para determinar as melhores para que um classificador seja capaz de diferenciar classes. Ao usar camadas convolucionais, caberá à rede encontrar as melhores características para solucionar o problema.

\section{Segmentação de Tumores Renais}
\label{sec:trabalhos-relacionados-tumores-renais}

Atualmente, vários estudos na literatura têm mostrado que o tumor renal é uma forma extremamente agressiva de câncer~\cite{muglia2015renal, dreisin2016treating, warren2018isup}, e os métodos computacionais têm sido de grande importância para ajudar a acelerar a detecção precoce. No trabalho de \citeonline{kaur2019hybrid} uma técnica de segmentação híbrida foi proposta com base em dois métodos que incluem \textit{Spatial Intuitionistic Fuzzy C-Means Clustering} (SIFCM) e \textit{Distance Regularized Level-Sets Evolution} (DRLSE) para extração de lesões renais em imagens de TC. Um conjunto de 40 TCs foi usado para validar o método e atingiu um Dice de 88,2\% para segmentação de lesões renais.

\citeonline{yang2020weakly} usaram um conjunto de CNNs fracamente supervisionada para segmentação de tumores renais. Para treinamento das CNNs, caixas delimitadoras de tumores renais foram fornecidas como anotações fracas. Com o intuito de melhorar o treinamento e a precisão da segmentação, propuseram duas fases de treinamento: em grupo e ponderado. No melhor cenário dos experimentos, apresentou resultado de 82,6\% de Dice usando um conjunto de 200 TCs.

No método proposto por \citeonline{RUAN2020101721} um modelo de rede adversária gerativa de compartilhamento de recursos multilateral (\textit{Multi-Branch Feature Sharing Generative Adversarial Network} - MB-FSGAN) foi proposto. Ele consiste em um extrator de características em várias escalas (\textit{Multi-Scale Feature Extractor} - MSFE), um localizador da área de interesse e uma rede adversária geradora de compartilhamento de características (\textit{Feature Sharing Generative Adversarial Network} - FSGAN). Os experimentos foram realizados em TCs de 113 pacientes com tumor renal. No melhor desempenho, obtiveram 85,9\% de Dice.

Por sua vez, \citeonline{Yan9098325} apresentaram um modelo denominado HybridNet 3D que consiste principalmente em duas partes: a rede de segmentação de primeiro plano e a rede de segmentação de \textit{PointCloud Sparse}. A rede de segmentação de primeiro plano tem uma estrutura semelhante à V-Net~\cite{v-net2016} e produz um mapa de segmentação de primeiro plano. A rede de segmentação de \textit{PointCloud Sparse} consiste em um módulo \textit{Voxel-to-Point} (VTP) e um módulo \textit{Sparse Segmentation Network} (SSN), onde o módulo SSN contém uma U-Net construída pela \textit{Submanifold Sparse Convolutional Networks} (SSCNs) em nuvens de pontos. A base de imagens usada consiste em 300 TCs, tendo seu método atingido 79,7\% de Dice para tumores renais.

%%--novo
\citeonline{turk2022kidney} desenvolveram um sistema para auxiliar os médicos na segmentação de tumores renais. O método proposto consistiu em melhorias nas fases de codificação e decodificação do modelo V-Net. Com a estrutura do bloco gargalo de duplo estágio (\textit{Two-Stage Bottleneck Block}), a arquitetura foi transformada em uma arquitetura única. Para avaliar o modelo, foram utilizadas 210 imagens de TC. O modelo desenvolvido obteve um Dice de 86,9\%, apresentando os melhores resultados em relação às redes comparadas.
%%--20 teste

%%--novo
\citeonline{Tanimoto22} apresentaram um método de aprendizado usando uma U-Net 3D, no qual erros em regiões implausíveis são suprimidos levando em consideração a estrutura global dos rins. No método proposto, adotou-se \textit{patches} cuboides para cobrir toda a estrutura no plano axial (transversal), que tende a ter uma simetria rotacional quando a região dos rins está localizada no centro de cada \textit{patch}. 213 imagens de TC foram usadas para avaliar o método proposto, obtendo-se 60,4\% de Dice.

Como pode ser analisado nos estudos supracitados, a tarefa de segmentar tumores renais é extremamente difícil, até mesmo para especialistas, pois não há uma distinção clara entre o contraste do tumor renal e a textura do rim, entre outras anormalidades, como os cistos. As abordagens híbrida e fracamente supervisionada \cite{kaur2019hybrid,yang2020weakly} minimizam a interferência manual de um médico especialista. Isso não é trivial, pois pode causar erros na marcação da região de interesse, uma vez que as técnicas propostas (SIFCM e \textit{convolutional conditional random fields}) necessitam de parâmetros adequados para cada tipo de imagem \cite{RUAN2020101721}. Consequentemente, erros de segmentação podem ser gerados e interferir no diagnóstico do médico especialista. Por fim, estudos como \citeonline{mehta2019segmenting}, \citeonline{Yan9098325}, \citeonline{turk2022kidney} e \citeonline{Tanimoto22} usam as redes totalmente conectadas para segmentação dos tumores renais, o que acabaram tendo bom desempenho devido às suas arquiteturas de alto desempenho.

%Existem outros métodos de aprendizado profundo bem-sucedidos para segmentar os tumores renais. Esses métodos destacam-se devido sua arquitetura 3D, que apresenta alto desempenho.
%Por sua vez, os métodos que usam a abordagem 3D \cite{jackson2018deep} têm algumas desvantagens, como o consumo de memória da unidade de processamento gráfico (GPU), alto custo computacional e a grande quantidade de conjunto de dados anotados totalmente 3D necessários.

\section{Segmentação de Rins e Tumores Renais}
\label{sec:trabalhos-relacionados-rins-e-tumores-renais}

Existem vários métodos de aprendizado profundo bem-sucedidos para segmentar rins e tumores renais em imagens de TC. Esses métodos destacam-se devido sua arquitetura 3D, que apresenta alto desempenho. No trabalho de \citeonline{yang2018automatic} é proposta uma rede 3D totalmente convolucional (\textit{Fully Convolutional Network} - FCN) que combina um módulo de pooling de pirâmide (\textit{Pyramid Pooling Module} - PPM) projetado para a segmentação de rins e tumores renais. Para extrair as regiões de interesse das imagens, usaram um método baseado em atlas. Os experimentos realizados em 140 pacientes de uma base de imagens privada, resultaram em um Dice de 93,1\% para rins e 80,2\% para tumores renais.

No estudo de \citeonline{turk2020kidney} foi proposto um modelo híbrido que usa recursos superiores dos modelos V-Nets existentes, considerando os estágios de codificação e decodificação separadamente. O codificador foi criado com base no modelo de fusão V-Net. O decodificador foi projetado com base na arquitetura ET-Net. Para validar o método proposto foi usado um conjunto de 210 tomografias. O modelo híbrido V-Net apresentou Dice médio de 97,7\% e 86,5\% para segmentação de rins e tumores renais, respectivamente.

\citeonline{QAYYUM2020104097} propuseram uma rede residual 3D híbrida com um \textit{Squeeze-and-Excitation} (SE) \textit{block} para segmentação volumétrica de rins e tumores renais. A rede proposta usa blocos SE para capturar informações espaciais com base na função de reponderação da rede residual. Um esquema de bloco residual sucessivo foi usado em cada estágio de codificador e decodificador dos blocos residuais 3D. O método foi avaliado em um conjunto de 210 pacientes e obteve resultados de 97,8\% de Dice para rins, e 86,8\% para tumores renais.

Para lidar com os desafios da segmentação renal e tumoral em imagens de TC, \citeonline{seru2020} propuseram um modelo denominado SE-ResNeXT U-Net (SERU), que combina as vantagens da SE-Net, ResNeXT e U-Net. Além disso, implementaram o modelo de maneira \textit{coarse-to-fine}\footnote{Segmentação \textit{coarse} (grosseira) é uma segmentação que não tem muitos detalhes. Segmentação \textit{fine} (fina) refere-se a uma segmentação com alto nível de detalhamento (ou seja, segmentação correta \textit{pixel} a \textit{pixel}).} para usar informações de contexto e fatias importantes dos rins esquerdo e direito. Um total de 300 tomografias foram usadas para realizar os experimentos. Os resultados apresentaram um Dice de 96,77\% para rins e 74,32\% para tumores renais.

\citeonline{ZHAO2020100357} apresentaram uma U-Net 3D supervisionada em várias escalas, o MSS U-Net (\textit{Multi-Scale Supervised} 3D U-Net) para segmentar rins e tumores renais. A arquitetura combina supervisão profunda com perda logarítmica exponencial para aumentar a eficiência do treinamento 3D U-Net. Além disso, introduziram um método de pós-processamento baseado em componentes conectados para aprimorar o desempenho do processo geral. Essa arquitetura apresentou Dice de 96,9\% para rins e 80,5\% para tumores renais, usando uma base de imagens de 210 pacientes.

%%-- novo 
\citeonline{9708025} propuseram uma técnica automatizada para delinear os rins e o tumor em imagens de TC usando a arquitetura \textit{Attention} U-Net. O modelo \textit{Attention} U-Net foi usado para dar maior ênfase às regiões de interesse (rins e tumores) e menos ênfase nas áreas que não estão em foco. O modelo foi avaliado em 205 imagens de TC (cinco TCs foram excluídas inicialmente) e obteve um Dice de 95,65\% e 93,86\% para rins e tumores renais, respectivamente. 
%%-- 20 teste

Outro trabalho que faz uso da abordagem 3D para segmentar rins e tumores renais foi apresentado por \citeonline{HELLER2021101821}. Os autores propuseram três arquiteturas inspiradas na U-Net 3D: uma simples, uma residual e uma residual de pré-ativação. Para medir a precisão do método proposto, um conjunto de 210 tomografias foi usado. Dentre todos os experimentos realizados, os melhores resultados foram obtidos usando a U-Net residual 3D, alcançando 97,4\% de Dice para rins e 85,1\% para tumores renais.

%%--novo
No estudo proposto por \citeonline{Lin2021} foram construídos dois modelos de segmentação 3D baseados em U-Net: um para segmentação automatizada do rim e outro para segmentação e detecção de tumor e cisto renal. O modelo de segmentação renal foi baseado em cascata de duas redes 3D U-Net. Para segmentar os tumores renais, uma única rede 3D U-Net foi construída. A avaliação do desempenho do modelo foi realizada por meio de uma base de imagens privada contendo 441 pacientes. O modelo proposto apresentou na segmentação de rins e tumores renais um Dice de 97,3\% e 84,4\%, respectivamente.
%%--66 de teste

%%--novo
\citeonline{YANG2022106616} propuseram uma nova rede neural profunda, a 3D \textit{Multi-Scale Residual Fully Convolutional Neural Network} (3D-MS-RFCNN) para melhorar a segmentação em tumores renais de tamanho grande. Para avaliar o desempenho do método proposto, uma base de imagens pública KiTS (210 imagens) foi usada para treinamento e um conjunto de dados hospitalares internos (270 imagens) para validação externa. Foram alcançados 91,62\% de Dice para rins e 71,64\% para tumores renais.
%--21 de teste

%%--novo
No trabalho de \citeonline{KANG2022103334} uma estrutura de segmentação em duas etapas: \textit{coarse} e \textit{fine}. Na etapa de segmentação \textit{coarse}, um método de treinamento prévio assistido por contorno é proposto para melhorar a capacidade da rede de aprender os contornos das bordas e extrair completamente a imagem da região de interesse contendo o rim e o tumor renal. No estágio de segmentação \textit{fine}, é proposta uma rede U-Net melhorada (LC-Unet) fundindo ConvLSTM e CNN 3D. Vários experimentos foram feitos para verificar o desempenho do método proposto usando 300 imagens de TC. No melhor resultado foram obtidos 96,39\% e 78,9\% de Dice para rins e tumores renais, respectivamente.
%%--90 de teste e data augmentation

%%--novo
%\citeonline{XUAN2022107360} desenvolveram o modelo \textit{Dynamic Graph Convolution} (DGC) para segmentação de rins e tumores renais. O DGC-Seg consiste em três componentes principais de aprendizagem. O primeiro componente é aprender representações de contexto, incluindo textura e recursos semânticos usando o codificador 3D nnU-Net. Em segundo, as características extraídas foram projetadas em estratégia de construção de grafos para associar características de imagem com nós de grafos e topologia. Em último lugar, o codificador e o decodificador baseados em convolução de grafos dinâmicos (DGC) são propostos para o raciocínio de relações dinâmicas de conexões complexas entre regiões de imagens. 210 imagens foram usadas para validar o modelo, resultando em um Dice de 96,1\% para rins e 92,6\% para tumores renais.
%%--42 de teste // data augmentation

Para segmentar os rins e tumores renais, observa-se que os métodos que utilizam abordagens 3D se destacaram devido ao seu alto desempenho. Em contrapartida, esse tipo de abordagem tem algumas desvantagens, como o consumo de memória da unidade de processamento gráfico (GPU - \textit{Graphics Processing Unit}), alto custo computacional e necessidade de grande quantidade de conjuntos de dados totalmente anotados em 3D. Assim, as abordagens 3D nem sempre são opções viáveis.

Em geral, os trabalhos relacionados descritos neste capítulo, apresentam diferentes abordagens e têm alcançado resultados cada vez mais expressivos para solucionar o problema da segmentação de rins e tumores renais. Foram vistos desde estudos que trabalham com abordagens que usam atlas \cite{wolz2013automated,yang2014automatic,yang2018automatic} a estudos que tem complexidade computacional, por meio de abordagens 3D~\cite{jackson2018deep,yang2018automatic,Yan9098325,QAYYUM2020104097,ZHAO2020100357,HELLER2021101821,Lin2021,Tanimoto22,YANG2022106616,KANG2022103334}. A Tabela~\ref{tab:trabalhos-relacionados} apresenta um resumo dos trabalhos relacionados descritos, na qual são apresentadas informações sobre as técnicas de segmentação, o número de pacientes e o desempenho Dice para rins e tumores.

Considerando as vantagens e limitações dos trabalhos relacionados, propõe-se um método eficiente para segmentar os rins e tumores renais em imagens de TC de pacientes doentes. De acordo com a literatura, os métodos que usam aprendizagem profunda superam as técnicas tradicionais em tarefas de visão computacional em geral \cite{krizhevsky2012imagenet}, e na segmentação de imagens de TC em particular \cite{christ2017automatic}.

Desse modo, o método proposto aborda três problemas: segmentação inicial dos rins, o que consequentemente reduz o espaço de busca para o diagnóstico de doenças renais; segmentação de tumores renais, a partir do resultado da segmentação dos rins; e a reconstrução dos tumores renais para aumentar a precisão das segmentações dos rins e tumores renais. Para isso, propõe-se o uso de duas redes de aprendizado profundo: a ResUnet que é capaz de segmentar e recuperar os tumores renais; e a DeepLabv3+ que é capaz de segmentar os tumores renais com precisão. E, finalmente, técnicas de processamento de imagem são usadas para reduzir falsos positivos.

\begin{table}[!ht]
\caption{Resumo dos trabalhos relacionados.}
\label{tab:trabalhos-relacionados}
\doublespacing
\centering
\resizebox{\columnwidth}{!}{
\begin{tabular}{ccp{8cm}cccc}
\cline{2-7} \parbox[t]{4mm}{\multirow{15}{*}{\rotatebox[origin=c]{90}{\textbf{Rins}}}}
                     & \textbf{Trabalho}                 & \centering \textbf{Técnica(s)}                                                & \textbf{Base}                                              & \textbf{Pacientes} & \textbf{Dice - Rim} & \textbf{Dice - Tumor} \\ \cline{2-7} 
                     & \citeonline{khalifa2011new}       & Modelo deformável baseado em \textit{Level Set}                    & Privada                                                    & 29                 & 95\%                & -                     \\
                     & \citeonline{wolz2013automated}    & Esquema hierárquico de registro e ponderação de atlas              & Privada                                                    & 150                & 92,5\%              & -                     \\
                     & \citeonline{Zhao2013ContextualIK} & Continuidade contextual na sequência de imagens de TC              & Privada                                                    & 10                 & 94,7\%              & -                     \\
                     & \citeonline{yang2014automatic}    & Registro de imagens multi-atlas                                    & Privada                                                    & 14 fatias          & 95,2\%              & -                     \\
                     & \citeonline{khalifa2016kidney}    & Estrutura híbrida baseada de modelos geométricos deformáveis e NMF & Privada                                                    & 36                 & 95\%                & -                     \\
                     & \citeonline{skalski2017kidney}    & Contorno ativo usando a estrutura \textit{Level Set}               & Privada                                                    & 10                 & 86,2\%              & -                     \\
\multicolumn{1}{l}{} & \citeonline{jackson2018deep}      & CNNs 3D                                                            & Privada                                                    & 113                & 88,5\%              & -                     \\
\multicolumn{1}{l}{} & \citeonline{mehta2019segmenting}  & Crowdsourcing e CNN                                                & Privada                                                    & 42                 & 93,2\%              & -                     \\
                     & \citeonline{9534007}              & U-Net, \textit{patch} e sistema de rastreamento                    & KiTS19                                                     & 270                & 90,63\%             & -                     \\ \cline{2-7} \parbox[t]{4mm}{\multirow{8}{*}{\rotatebox[origin=c]{90}{\textbf{Tumores}}}}
                     & \citeonline{kaur2019hybrid}       & Estrutura hibrida baseada em SIFCM e DRLSE                         & Privada                                                    & 40                 & -                   & 88,2\%                \\
                     & \citeonline{yang2020weakly}       & CNNs fracamente supervisionada                                     & Privada                                                    & 200                & -                   & 82,6\%                \\
                     & \citeonline{RUAN2020101721}       & MB-FSGAN                                                           & Privada                                                    & 113                & -                   & 85,9\%                \\
\multicolumn{1}{l}{} & \citeonline{Yan9098325}           & HybridNet 3D                                                       & KiTS19                                                     & 300                & -                   & 79,7\%                \\
\multicolumn{1}{l}{} & \citeonline{turk2022kidney}       & V-Net usando \textit{Two-Stage Bottleneck Block}                   & KiTS19                                                     & 210                & -                   & 86,9\%                \\
                     & \citeonline{Tanimoto22}           & 3D U-Net preservando a simetria rotacional em cortes axiais        & Privada                                                    & 213                & -                   & 60,4\%                \\ \cline{2-7} \parbox[t]{4mm}{\multirow{12}{*}{\rotatebox[origin=c]{90}{\textbf{Rins e Tumores}}}}
                     & \citeonline{yang2018automatic}    & Rede 3D com PPM + Atlas                                            & Privada                                                    & 140                & 93,1\%              & 80,2\%                \\
                     & \citeonline{turk2020kidney}       & Modelo hibrido V-Net                                               & KiTS19                                                     & 210                & 97,7\%              & 86,5\%                \\
                     & \citeonline{QAYYUM2020104097}     & Rede residual híbrida 3D com SE                                    & KiTS19                                                     & 210                & 97,8\%              & 86,8\%                \\
\multicolumn{1}{l}{} & \citeonline{seru2020}             & SE-ResNeXT U-Net                                                   & KiTS19                                                     & 300                & 96,77\%             & 74,32\%               \\
\multicolumn{1}{l}{} & \citeonline{ZHAO2020100357}       & MSS U-Net 3D                                                       & KiTS19                                                     & 210                & 96,9\%              & 80,5\%                \\
\multicolumn{1}{l}{} & \citeonline{9708025}              & \textit{Attention} U-Net                                                    & KiTS19                                                     & 205                & 95,65\%             & 93,86\%               \\
\multicolumn{1}{l}{} & \citeonline{HELLER2021101821}     & U-Nets 3D: simples, residual e residual de pré-ativação            & KiTS19                                                     & 210                & 97,4\%              & 85,1\%                \\
\multicolumn{1}{l}{} & \citeonline{Lin2021}              & 3D U-Net                                                           & Privada                                                    & 441                & 97,3\%              & 84,4\%                \\
\multicolumn{1}{l}{} & \citeonline{YANG2022106616}       & 3D-MS-RFCNN                                                        & \begin{tabular}[c]{@{}c@{}}KiTS19 +\\ Privada\end{tabular} & 480                & 91,62\%             & 71,64\%               \\
                     & \citeonline{KANG2022103334}       & 3D-CNN e ConvLSTM                                                  & KiTS19                                                     & 300                & 96,39\%             & 78,9\%                \\ \cline{2-7} 
\end{tabular}
}
\end{table}

\section{Considerações Finais}

Neste capítulo, foi feita uma revisão das produções científicas relevantes relacionadas à segmentação de rins e tumores renais em imagens de TC. Os métodos e os resultados obtidos nestes trabalhos foram brevemente apresentados. Além disso, foram feitas considerações sobre as diferenças entre as diferentes abordagens apresentadas, a fim de obter um panorama do que vem sendo produzido pela comunidade científica em relação a esse tema.

No próximo capítulo, serão apresentados os conceitos teóricos fundamentais para o desenvolvimento deste trabalho.